\documentclass[10pt, a4paper, nottitlepage]{article}

% Include packages 
\usepackage[left=2.5cm,right=2.5cm,top=2.5cm,bottom=2.5cm]{geometry}
\usepackage[utf8]{inputenc}
\usepackage[T1]{fontenc}
\usepackage[english]{babel}

\usepackage{float}
\usepackage{gensymb}
\usepackage{xcolor}
\usepackage{tikz}
\usepackage{graphicx}
\usepackage{lmodern}
\usepackage{amsmath,amsfonts,amssymb}
\usepackage{mathtools}
\usepackage{booktabs}
\usepackage{fancyhdr}
\usepackage{pdfpages}
\usepackage{extarrows}
\usepackage{subcaption}
\usepackage{url}
\usepackage{hyperref}
\usepackage{fontawesome}
\usepackage{marvosym}
\usepackage{acronym}
\usepackage{enumitem}
\usepackage{pdflscape}
\usepackage{listings}
\usepackage{siunitx}
%\usepackage{icomma}
\usepackage{bigdelim}
\usepackage{isotope}
\setlength{\parindent}{0em} 
\newcommand{\IU}{\mathrm{i}} 

\renewcommand{\lstlistingname}{Python-Code}
\lstset{
	language=Python,
	frame=single,
	morekeywords={as,True,False},
	basicstyle=\footnotesize,
	keywordstyle=\color{blue}\bfseries,
	commentstyle=\color{red},
	stringstyle=\color{red},
	showstringspaces=false,
	showspaces=false,
	numbers=left,
	extendedchars=true,
	literate={ö}{{\"o}}1 {Ö}{{\"O}}1 {ä}{{\"a}}1 {Ä}{{\"A}}1 {ü}{{\"u}}1 {Ü}{{\"U}}1 {ß}{{\ss}}1}

\title{Introduction to Computational Physics  -- Exercise 8}
\author{Simon Groß-Bölting, Lorenz Vogel, Sebastian Willenberg}
\date{June 19, 2020}

\setcounter{section}{1}
% The beginning of the document...
\begin{document}
\maketitle


\section*{Stability Analysis of Many Species Population Dynamics}
In this exercise we study the evolution of $6$ populations according to the following equations for population dynamics: $3$ predator- ($P_i$) and $3$ prey-species ($N_i$), all parameters positive, always $i,j=1,\,\dots,\,3$:
\begin{equation}
	\frac{\mathrm{d}N_i}{\mathrm{d}t} = N_i\biggl(a_i-N_i-\sum_{j}b_{ij}P_j\biggr)
	\qquad\quad\text{and}\quad\qquad
	\frac{\mathrm{d}P_i}{\mathrm{d}t} = P_i\biggl(\sum_{j}c_{ij}N_j-d_i\biggr)
	\label{eq:population-model}
\end{equation}
The parameters chosen are $a_1=56$, $a_2=12$, $a_3=35$, $d_1=85$, $d_2=9$ and $d_3=35$; the parameters $b_{ij}$ and $c_{ij}$ are given in matrix form here:
\begin{equation}
	b_{ij} = \begin{pmatrix}20&30&5\\1&3&7\\4&10&20\end{pmatrix}
	\qquad\quad\text{and}\quad\qquad
	c_{ij} = \begin{pmatrix}20&30&35\\3&3&3\\7&8&20\end{pmatrix}
\end{equation}
\begin{description}
	\item[Notices:] The unusual feature here in the equations is that the prey populations $N_i$ have a Verhulst style growth limiting factor in their equations, which limits their growth even if there is no predator (model for limited resources even in absence of predators).
	\begin{itemize}
		\item Please do not try to make the equations dimensionless, just use the numbers given here.
		\item In this mathematical example the solutions can become negative in some cases, which is unphysical for population dynamics. Nevertheless let us please use this case to show the mathematical features.
	\end{itemize}
\end{description}




\begin{description}
	\item[Task:] What are the fixed points (FP) for the system of equations given above? \textit{Hint 1:} No complicated computations are necessary, the idea is that you should guess the fixed points very easily. Compare our previous examples.
	\textit{Hint 2:}  This time there are three fixed points! In addition to our ``usual'' ones, there is a third one related to the Verhulst growth limiting factor in the first three equations.
\end{description}

First of all, we abbreviate the functions occurring in Eq. \eqref{eq:population-model} with $n_i(N_i,\,\vec{P})$ and $p_i(P_i,\,\vec{N})$:
\begin{align}
	n_i(N_i,\,\vec{P}) 
	&:= \frac{\mathrm{d}N_i}{\mathrm{d}t} 
	= N_i\biggl(a_i-N_i-\sum_{j}b_{ij}P_j\biggr)
	&&\text{with}
	&\vec{P}=\begin{pmatrix}P_1&P_2&P_3\end{pmatrix}^T \\
	p_i(P_i,\,\vec{N})
	&:= \frac{\mathrm{d}P_i}{\mathrm{d}t} 
	= P_i\biggl(\sum_{j}c_{ij}N_j-d_i\biggr)
	&&\text{with}
	&\vec{N}=\begin{pmatrix}N_1&N_2&N_3\end{pmatrix}^T
\end{align}
We now define the vectorial function $\vec{f}(\vec{N},\,\vec{P})$ for arguments 
$\vec{N}$ (prey-species) and $\vec{P}$ (predator-species):
\begin{equation}
	\vec{f}(\vec{N},\,\vec{P}) =
	\begin{pmatrix}\dot{\vec{N}}\\\dot{\vec{P}}\end{pmatrix}
	\qquad\text{with}\qquad
	\dot{\vec{N}} 
	= \frac{\mathrm{d}\vec{N}}{\mathrm{d}t}
	= 	\begin{pmatrix} 
			n_1(N_1,\,\vec{P}) \\ n_2(N_2,\,\vec{P}) \\ n_3(N_3,\,\vec{P})
		\end{pmatrix}
	\quad\text{and}\quad
	\dot{\vec{P}} 
	= \frac{\mathrm{d}\vec{P}}{\mathrm{d}t}
	=	\begin{pmatrix}
			p_1(P_1,\,\vec{N}) \\ p_2(P_2,\,\vec{N}) \\ p_3(P_3,\,\vec{N})
		\end{pmatrix}
\end{equation}
The condition for fixed points (FP) in the case of interacting populations (multi-dimensional case) is more or less analogous to the one-dimensional case, but in vector form:
\begin{equation}
	\text{$\begin{pmatrix}\vec{N}^*\\\vec{P}^*\end{pmatrix}$ fixed point}
	\quad\Longleftrightarrow\quad
	\vec{f}(\vec{N}^*,\,\vec{P}^*) = \vec{0}
	\quad\Longleftrightarrow\quad
	\frac{\mathrm{d}\vec{N}}{\mathrm{d}t}\Big\vert_{\vec{N}=\vec{N}^*} = \vec{0}
	\quad\land\quad
	\frac{\mathrm{d}\vec{P}}{\mathrm{d}t}\Big\vert_{\vec{P}=\vec{P}^*}  = \vec{0}
\end{equation}

If we use the values given by the matrices $b_{ij}$ and $c_{ij}$, the function $\vec{f}(\vec{N},\,\vec{P})$ is given by the following system of equations:
\begin{align}
	\vec{f}(\vec{N},\,\vec{P}) =
	\begin{pmatrix}\dot{\vec{N}}\\\dot{\vec{P}}\end{pmatrix}
	\qquad\Longleftrightarrow\qquad
	\left\{\begin{array}{l}
		\dot{N_1} = N_1\left(56-N_1-20P_1-30P_2-5P_3\right) \\
		\dot{N_2} = N_2\left(12-N_2-P_1-3P_2-7P_3\right) \\
		\dot{N_3} = N_3\left(35-N_3-4P_1-10P_2-20P_3\right) \\
		\dot{P_1} = P_1\left(20N_1+30N_2+35N_3-85\right) \\
		\dot{P_2} = P_2\left(3N_1+3N_2+3N_3-9\right) \\
		\dot{P_3} = P_3\left(7N_1+8N_2+10N_3-35\right) 
	\end{array}\right.
\end{align}

If we look at Eq. \eqref{eq:population-model}, the \textbf{first fixed point} is obviously given by $N_i^*=P_i^*=0$, or in vector form:
\begin{equation}
	N_i^* = P_i^* = 0
	\qquad\Longleftrightarrow\qquad
	\begin{pmatrix}\vec{N}^*&\vec{P}^*\end{pmatrix}^T
	= \begin{pmatrix}0&0&0&0&0&0\end{pmatrix}^T
\end{equation}

We get the second fixed point by first looking at for which vector $\vec{N}^*$ the system of equations $\dot{\vec{P}}$ becomes zero (except for $\vec{P}^*=0$):
\begin{align}
	d_i = \sum_{j}c_{ij}N_j^*
	\quad\Longrightarrow\quad
	\left\{\begin{array}{l}
		85 = 20N_1^*+30N_2^*+35N_3^* \\
		9 = 3N_1^*+3N_2^*+3N_3^* \\
		35 = 7N_1^*+8N_2^*+10N_3^*
	\end{array}\right\}
	\quad\Longrightarrow\quad
	N_j^* = 1
\end{align}
With the condition $\vec{N}^* = 1$ we get for the components of $\vec{P}^*$:
\begin{align}
	N_i^* = a_i - \sum_{j}b_{ij}P_j^*
	\quad\xLongrightarrow[]{\text{\,$N_i^*=1$\:\:}}\quad
	\left\{\begin{array}{l}
		1 = 56-\left(20P_1^*+30P_2^*+5P_3^*\right) \\
		1 = 12-\left(P_1^*+3P_2^*+7P_3^*\right) \\
		1 = 35-\left(4P_1^*+10P_2^*+20P_3^*\right)
	\end{array}\right\}
	\quad\Longrightarrow\quad
	P_j^* = 1
	\label{eq:fixed-point-two-condition}
\end{align}
The \textbf{second fixed point} is therefore given by $N_i^*=P_i^*=1$, or in vector form:
\begin{equation}
	N_i^* = P_i^* = 1
	\qquad\Longleftrightarrow\qquad
	\begin{pmatrix}\vec{N}^*&\vec{P}^*\end{pmatrix}^T
	= \begin{pmatrix}1&1&1&1&1&1\end{pmatrix}^T
\end{equation}
We have now received our two ``usual'' results for fixed points ($N_i^*=P_i^*=0$ and $N_i^*=P_i^*=1$).
By setting $\vec{P}^*=0$ and looking at the condition in Eq. \eqref{eq:fixed-point-two-condition} again, we get the corresponding vector $\vec{N}^*$:
\begin{equation}
	\vec{P}^* = \begin{pmatrix}P_1^*\\P_2^*\\P_3^*\end{pmatrix}
	= \begin{pmatrix}0\\0\\0\end{pmatrix}
	\quad\Longrightarrow\quad
	N_i^* = a_i - \sum_{j}b_{ij}P_j^* = a_i
	\quad\Longrightarrow\quad
	\vec{N}^* = \begin{pmatrix}N_1^*\\N_2^*\\N_3^*\end{pmatrix} 
	= \begin{pmatrix}56\\12\\35\end{pmatrix}
\end{equation}
Our result for the \textbf{third fixed point} is:
\begin{equation}
	N_i^* = a_i \quad\land\quad P_i^* = 0
	\qquad\Longleftrightarrow\qquad
	\begin{pmatrix}\vec{N}^*&\vec{P}^*\end{pmatrix}^T
	= \begin{pmatrix}56&12&35&0&0&0\end{pmatrix}^T
\end{equation}



\begin{description}
	\item[Task:] What is the Jacobi matrix $\mathbf{A}$ at the non-trivial fixed point? (non-trivial FP means here that \textit{all} elements are unequal to zero. There is only one FP with this property)
\end{description}

Jacobi matrix:
\begin{equation}
	\mathrm{D}\vec{f} =
	\begin{pmatrix}
		\frac{\partial n_1}{\partial N_1}&0&0&
		\frac{\partial n_1}{\partial P_1}&
		\frac{\partial n_1}{\partial P_2}&
		\frac{\partial n_1}{\partial P_3} \\
		0&\frac{\partial n_2}{\partial N_2}&0&
		\frac{\partial n_2}{\partial P_1}&
		\frac{\partial n_2}{\partial P_2}&
		\frac{\partial n_2}{\partial P_3} \\
		0&0&\frac{\partial n_3}{\partial N_3}&
		\frac{\partial n_3}{\partial P_1}&
		\frac{\partial n_3}{\partial P_2}&
		\frac{\partial n_3}{\partial P_3} \\
		\frac{\partial p_1}{\partial N_1}&
		\frac{\partial p_1}{\partial N_2}&
		\frac{\partial p_1}{\partial N_3}&
		\frac{\partial p_1}{\partial P_1}&0&0 \\
		\frac{\partial p_2}{\partial N_1}&
		\frac{\partial p_2}{\partial N_2}&
		\frac{\partial p_2}{\partial N_3}&
		0&\frac{\partial p_2}{\partial P_2}&0 \\
		\frac{\partial p_3}{\partial N_1}&
		\frac{\partial p_3}{\partial N_2}&
		\frac{\partial p_3}{\partial N_3}&
		0&0&\frac{\partial p_3}{\partial P_3} \\
	\end{pmatrix}
\end{equation}

\begin{equation}
	\frac{\partial n_i}{\partial N_i}
	= \frac{\partial}{\partial N_i}
	\Biggl[N_i\biggl(a_i-N_i-\sum_{j}b_{ij}P_j\biggr)\Biggr]
	= \biggl(a_i-N_i-\sum_{j}b_{ij}P_j\biggr)-N_i
\end{equation}
\begin{equation}
	\frac{\partial p_i}{\partial P_i}
	= \frac{\partial}{\partial P_i}
	\Biggl[P_i\biggl(\sum_{j}c_{ij}N_j-d_i\biggr)\Biggr]
	= \sum_{j}c_{ij}N_j-d_i
\end{equation}
\begin{equation}
	\frac{\partial n_i}{\partial P_j}
	= \frac{\partial}{\partial P_j}
	\Biggl[N_i\biggl(a_i-N_i-\sum_{j}b_{ij}P_j\biggr)\Biggr]
	= -N_ib_{ij}
\end{equation}
\begin{equation}
	\frac{\partial p_i}{\partial N_j}
	= \frac{\partial}{\partial N_j}
	\Biggl[P_i\biggl(\sum_{j}c_{ij}N_j-d_i\biggr)\Biggr]
	= P_ic_{ij}
\end{equation}

\begin{equation}
	\mathbf{A} := \mathrm{D}\vec{f}(N_i^*=P_i^*=1) =
	\begin{pmatrix}
		-1&0&0&-20&-30&-5 \\
		0&-1&0&-1&-3&-7 \\
		0&0&-1&-4&-10&-20 \\
		20&30&35&0&0&0 \\
		3&3&3&0&0&0 \\
		7&8&20&0&0&0 \\
	\end{pmatrix}
\end{equation}



\begin{description}
	\item[Task:] Determine the eigenvalues $\lambda_i$ and eigenvectors $\mathbf{v}_i$ ($i=1,\,\dots,\,6$) of $\mathbf{A}$ for this fixed point. Choose an initial state
	\begin{equation}
		\mathbf{n}_0 = \sum_{i=1}^{6} c_i\mathbf{v}_i
		\qquad\text{with}\qquad
		c_1 = c_2 = 3\,,\quad
		c_3 = c_4 = 1\,,\quad
		c_5 = -5\,,\quad
		c_6 = 0.1
	\end{equation}
	Plot and discuss the time-dependent evolution of the six populations (given by this linear
	model); in particular what about oscillations, growth or decay?
\end{description}

\begin{table}[h!]
	\centering
	\renewcommand{\arraystretch}{1.2}
	\caption{Eigenvalues and eigenvectors}
	\vspace{-0.2cm}
	\begin{tabular}{ll}
		\toprule
		eigenvalues $\lambda_i$ & eigenvectors $\mathbf{v}_i$ \\
		\midrule
		$-0.5+33.626\IU$ & \\
		$-0.5-33.626\IU$ & \\
		$-0.5+7.679\IU$ & \\
		$-0.5-7.679\IU$ & \\
		$-1.136$ & \\
		$+0.136$ & \\
		\bottomrule
	\end{tabular}
	\label{tab:stationary-points}
\end{table}

If $\lambda_i$ is an eigenvalue of $\mathbf{A}$ with the eigenvector $\mathbf{v}_i$, then
$\exp(\lambda_i t)$ is eigenvalue of $\exp(\mathbf{A}t)$.

\begin{equation}
	\mathbf{n}(t) = \exp(\mathbf{A}t)\mathbf{n}_0
	\qquad\text{with}\qquad
	\mathbf{A} = \mathrm{D}\vec{f}
\end{equation}
$\mathbf{n}_0$ is our initial condition. 
\begin{equation}
	\mathbf{n}(t)
	= \exp(\mathbf{A}t)\sum_{i=1}^{6}c_i\mathbf{v}_i
	= \sum_{i=1}^{6}c_i\exp(\mathbf{A}t)\mathbf{v}_i
	= \sum_{i=1}^{6}c_i\exp(\lambda_i t)\mathbf{v}_i
\end{equation}
We have an oscillating solution.


\begin{figure}[t!]
	\centering
	\begin{subfigure}[t]{0.49\textwidth}
		\centering
		\includegraphics[width=\textwidth]{figures/Population-Time-Evolution-01.pdf}
		\caption{}
		\label{fig:species-1}
	\end{subfigure}
	\begin{subfigure}[t]{0.49\textwidth}
		\centering
		\includegraphics[width=\textwidth]{figures/Population-Time-Evolution-02.pdf}
		\caption{}
		\label{fig:species-2}
	\end{subfigure}
	
	\begin{subfigure}[t]{0.49\textwidth}
		\centering
		\includegraphics[width=\textwidth]{figures/Population-Time-Evolution-03.pdf}
		\caption{}
		\label{fig:species-3}
	\end{subfigure}
	\caption{Time-dependent evolution of the six populations}
	\label{fig:time-evolution}
\end{figure}

\newpage
\lstinputlisting[caption={Numerical determination of the time-dependent evolution of the six populations for a given initial state using the eigenvalues and the eigenvectors},label={py:code}]{Homework08_Many-Species-Population-Dynamics_Gross-Boelting_Vogel_Willenberg.py}
\end{document}
