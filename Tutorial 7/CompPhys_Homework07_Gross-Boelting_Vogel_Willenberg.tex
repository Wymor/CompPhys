\documentclass[10pt, a4paper, nottitlepage]{article}

% Include packages 
\usepackage[left=2.5cm,right=2.5cm,top=2.5cm,bottom=2.5cm]{geometry}
\usepackage[utf8]{inputenc}
\usepackage[T1]{fontenc}
\usepackage[english]{babel}

\usepackage{float}
\usepackage{gensymb}
\usepackage{xcolor}
\usepackage{tikz}
\usepackage{graphicx}
\usepackage{lmodern}
\usepackage{amsmath,amsfonts,amssymb}
\usepackage{mathtools}
\usepackage{booktabs}
\usepackage{fancyhdr}
\usepackage{pdfpages}
\usepackage{extarrows}
\usepackage{subcaption}
\usepackage{url}
\usepackage{hyperref}
\usepackage{fontawesome}
\usepackage{marvosym}
\usepackage{acronym}
\usepackage{enumitem}
\usepackage{pdflscape}
\usepackage{listings}
\usepackage{siunitx}
\usepackage{icomma}
\usepackage{bigdelim}
\usepackage{isotope}
\setlength{\parindent}{0em} 

\renewcommand{\lstlistingname}{Python-Code}
\lstset{
	language=Python,
	frame=single,
	morekeywords={as,True,False},
	basicstyle=\footnotesize,
	keywordstyle=\color{blue}\bfseries,
	commentstyle=\color{red},
	stringstyle=\color{red},
	showstringspaces=false,
	showspaces=false,
	numbers=left,
	extendedchars=true,
	literate={ö}{{\"o}}1 {Ö}{{\"O}}1 {ä}{{\"a}}1 {Ä}{{\"A}}1 {ü}{{\"u}}1 {Ü}{{\"U}}1 {ß}{{\ss}}1}

\title{Introduction to Computational Physics  -- Exercise 7}
\author{Simon Groß-Bölting, Lorenz Vogel, Sebastian Willenberg}
\date{June 12, 2020}

\setcounter{section}{1}
% The beginning of the document...
\begin{document}
\maketitle

\section*{Population dynamics}
The simple equation of growth of a population, as proposed by Malthus (1798), has been improved by Verhulst (1836) including a growth limiting term, which represents the finite amount of
resources available.
In this exercise we study a modified form of Verhulst's equation for population dynamics:
\begin{equation}
	\frac{\mathrm{d}N}{\mathrm{d}t} = rN(1-N/K)-\frac{BN^2}{A^2+N^2}
\end{equation}
where all parameters $r$, $K$, $A$ and $B$ are positive. It is a more complex example, in which
the growth behaviour depends on whether $N$ is smaller or larger than a critical populations
size $A$.

\subsection*{Dimensional analysis}
\begin{description}
	\item[Task:] Determine the dimension of the parameters and rewrite the equation in dimensionless form. Note that there are different possibilities.
	Please formulate a dimensionless time $\tau$ that is not defined on the basis of $r$.
	Use $n=N/A$ as the dimensionless version of $N$.
\end{description}

\begin{table}[h!]
	\centering
	\renewcommand{\arraystretch}{1.2}
	\caption{Dimension of the parameters}
	\vspace{-0.2cm}
	\begin{tabular}{rcl}
		\toprule
		description & 
		parameter &
		dimension \\
		\midrule
		time & $t$ & $[t] = \si{\second}$ \\
		population number & $N$ & $[N] = 1$ \\
		growth rate & $r$ & $[r] = \si{\per\second}$ \\
		growth limiting number & $K$ & $[K] = 1$ \\
		critical population size & $A$ & $[A] = 1$ \\
		--- & $B$ & $[B] = \si{\per\second}$ \\
		\bottomrule
	\end{tabular}
	\label{tab:energy-eigenvalues}
\end{table}


By dimensional analysis we can renormalize the variables, such that the above equation for population dynamics becomes dimensionless: The steps required for this are well described in the Wikipedia article ``\href{https://en.wikipedia.org/wiki/Nondimensionalization}{Nondimensionalization}''.
First of all we have to identify all the independent and dependent variables in the given equation:
\begin{equation}
	\frac{\mathrm{d}N}{\mathrm{d}t} 
	= rN\left(1-\frac{N}{K}\right)-\frac{BN^2}{A^2+N^2}
	= rN - \frac{rN^2}{K}-\frac{BN^2}{A^2+N^2}
\end{equation}
In this equation the independent variable is $t$ (time) and the dependent variable is $N = N(t)$ (population number).
Now we replace each of the two variables with the product of a dimensionless variable ($\tau$ and $n$) and a characteristic unit ($N_\mathrm{c}$ and $t_\mathrm{c}$):
\begin{equation}
	N := N_\mathrm{c}n \quad\Leftrightarrow\quad n := N/N_\mathrm{c}
	\qquad\qquad\text{and}\qquad\qquad
	t := t_\mathrm{c}\tau \quad\Leftrightarrow\quad \tau := t/t_\mathrm{c}
\end{equation}

The subscripted ``$\mathrm{c}$'' added to a variable-name is used to denote the characteristic unit used to scale that variable.
As required in the task, we set $N_\mathrm{c}:=A$, so that we use $n:=N/A$ as the dimensionless version of the population number $N$.
Then we obtain the following equation:
\begin{equation}
	\frac{A}{t_\mathrm{c}}\,\frac{\mathrm{d}n}{\mathrm{d}\tau}
	= rAn - \frac{rA^2}{K}\,n^2 - \frac{BA^2n^2}{A^2+A^2n^2} 
	= rAn - \frac{rA^2}{K}\,n^2 - B\,\frac{n^2}{1+n^2}
\end{equation}

Dividing the entire equation by the coefficient $A/t_\mathrm{c}$ in front of the first derivative term and using the relation
\begin{equation}
	\frac{n^2}{1+n^2}
	= \left[\frac{1+n^2}{n^2}\right]^{-1}
	= \left[1+\frac{1}{n^2}\right]^{-1}
	= \left[1+n^{-2}\right]^{-1}
	= \frac{1}{1+n^{-2}}
\end{equation}
gives us:
\begin{equation}
	\frac{\mathrm{d}n}{\mathrm{d}\tau}
	= t_\mathrm{c}rn - \frac{t_\mathrm{c}rA}{K}\,n^2 - \frac{t_\mathrm{c}B}{A}\,\frac{1}{1+n^{-2}}
\end{equation}

Now the characteristic unit for each variable can be defined, such that the coefficients of as many terms as possible become $1$.
Since we have already determined the dimensionless version of $N$, we only have to define the characteristic unit $t_\mathrm{c}$ of the time $t$.
Because we are supposed to formulate a dimensionless time $\tau = t/t_\mathrm{c}$ that is not defined on the basis of the growth rate $r$, we have to use the coefficient in front of the last term:
\begin{equation}
	\frac{t_\mathrm{c}B}{A} \overset{\text{!}}{=} 1
	\qquad\Longrightarrow\qquad
	t_\mathrm{c} := \frac{A}{B}
\end{equation}
Now we can rewrite the given equation for population dynamics in its dimensionless form:
\begin{equation}
	\frac{\mathrm{d}n}{\mathrm{d}\tau}
	= \frac{rA}{B}\,n - \frac{rA^2}{BK}\,n^2 - \frac{1}{1+n^{-2}}
\end{equation}
By defining the dimensionless parameter $D := rA/B$, we can further simplify the dimensionless equation:
\begin{equation}
	\frac{\mathrm{d}n}{\mathrm{d}\tau}
	= Dn - \frac{DA}{K}\,n^2 - \frac{1}{1+n^{-2}}
	\qquad\text{with}\qquad
	D := \frac{rA}{B}
\end{equation}

\subsection*{Stationary points}
\begin{description}
	\item[Task:] Determine the stationary points $n^*$ for $K/A=\num{7.3}$. Note that for $n^*\neq 0$ these values are solutions of a cubic equation; it depends on $n$ and the remaining free parameter. The cubic equation should be derived by yourself analytically; its zero points you can obtain numerically / graphically by using e.g. Mathematica. When do one or three real solutions exist as a function of the remaining free parameter?
	(Hint: we do not ask for some analytical formula here! It is enough to vary the free parameter and check using Mathematica which three solutions for the stationary points you get; as stationary points only real solutions are valid. Only one digit after the comma is enough, in other words you vary the free parameter by about $\num{0.05}$.).\\
	Which of the stationary points is stable and unstable?
\end{description}



\begin{description}
	\item[Definition:]  A fixed point (FP) or stationary point of a differentiable function of one variable is a point on the graph of the function where the function's derivative is zero.
\end{description}

\begin{equation}
	\text{$n=n^*$ stationary point of $n=n(\tau)$}
	\qquad\quad\Longleftrightarrow\quad\qquad
	\frac{\mathrm{d}n}{\mathrm{d}\tau}\Big\vert_{n=n^*} = 0	
\end{equation}

%\lstinputlisting[caption={Numerical solution of the Schrödinger equation using the Numerov method},label={py:code}]{Numerov-Algorithm_Homework04_Gross-Boelting_Vogel_Willenberg.py}
 
\end{document}
