\documentclass[10pt, a4paper, nottitlepage]{article}

% Include packages 
\usepackage[left=2.5cm,right=2.5cm,top=2.5cm,bottom=2.5cm]{geometry}
\usepackage[utf8]{inputenc}
\usepackage[T1]{fontenc}
\usepackage[english]{babel}

\usepackage{float}
\usepackage{gensymb}
\usepackage{xcolor}
\usepackage{tikz}
\usepackage{graphicx}
\usepackage{lmodern}
\usepackage{amsmath,amsfonts,amssymb}
\usepackage{mathtools}
\usepackage{booktabs}
\usepackage{fancyhdr}
\usepackage{pdfpages}
\usepackage{extarrows}
\usepackage{subcaption}
\usepackage{url}
\usepackage{hyperref}
\usepackage{fontawesome}
\usepackage{marvosym}
\usepackage{acronym}
\usepackage{enumitem}
\usepackage{pdflscape}
\usepackage{listings}
\usepackage{siunitx}
\usepackage{icomma}
\usepackage{bigdelim}
\usepackage{isotope}
\setlength{\parindent}{0em} 

\renewcommand{\lstlistingname}{Python-Code}
\lstset{
	language=Python,
	frame=single,
	morekeywords={as,True,False},
	basicstyle=\footnotesize,
	keywordstyle=\color{blue}\bfseries,
	commentstyle=\color{red},
	stringstyle=\color{red},
	showstringspaces=false,
	showspaces=false,
	numbers=left,
	extendedchars=true,
	literate={ö}{{\"o}}1 {Ö}{{\"O}}1 {ä}{{\"a}}1 {Ä}{{\"A}}1 {ü}{{\"u}}1 {Ü}{{\"U}}1 {ß}{{\ss}}1}

\title{Introduction to Computational Physics  -- Exercise 7}
\author{Simon Groß-Bölting, Lorenz Vogel, Sebastian Willenberg}
\date{June 12, 2020}

\setcounter{section}{1}
% The beginning of the document...
\begin{document}
\maketitle

\section*{Population dynamics}
In this exercise we study the following equation for population dynamics:
\begin{equation}
	\frac{\mathrm{d}N}{\mathrm{d}t} = rN(1-N/K)-\frac{BN^2}{A^2+N^2}
\end{equation}
where all parameters $r$, $K$, $A$ and $B$ are positive. It is a more complex example, in which
the growth behaviour depends on whether $N$ is smaller or larger than a critical populations
size $A$.

\subsection*{Dimensional analysis}
Determine the dimension of the parameters and rewrite the equation in dimensionless form. Note that there are different possibilities.
Please formulate a dimensionless time $\tau$ that is not defined on the basis of $r$.
Use $n=N/A$ as the dimensionless version of $N$.

\subsection*{Stationary points}
Determine the stationary points $n^*$ for $K/A=\num{7.3}$. Note that for $n^*\neq 0$
these values are solutions of a cubic equation; it depends on $n$ and the remaining free
parameter. The cubic equation should be derived by yourself analytically; its zero
points you can obtain numerically / graphically by using e.g. Mathematica.
When do one or three real solutions exist as a function of the remaining free parameter?
(Hint: we do not ask for some analytical formula here! It is enough to vary the free
parameter and check using Mathematica which three solutions for the stationary
points you get; as stationary points only real solutions are valid. Only one digit
after the comma is enough, in other words you vary the free parameter by about
$\num{0.05}$.). Which of the stationary points is stable and unstable?

%\lstinputlisting[caption={Numerical solution of the Schrödinger equation using the Numerov method},label={py:code}]{Numerov-Algorithm_Homework04_Gross-Boelting_Vogel_Willenberg.py}
 
\end{document}
