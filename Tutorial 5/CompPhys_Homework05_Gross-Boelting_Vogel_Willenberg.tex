\documentclass[10pt, a4paper, nottitlepage]{article}

% Include packages 
\usepackage[left=2.5cm,right=2.5cm,top=2.5cm,bottom=2.5cm]{geometry}
\usepackage[utf8]{inputenc}
\usepackage[T1]{fontenc}
\usepackage[english]{babel}

\usepackage{float}
\usepackage{gensymb}
\usepackage{xcolor}
\usepackage{tikz}
\usepackage{graphicx}
\usepackage{lmodern}
\usepackage{amsmath,amsfonts,amssymb}
\usepackage{mathtools}
\usepackage{booktabs}
\usepackage{fancyhdr}
\usepackage{pdfpages}
\usepackage{extarrows}
\usepackage{subcaption}
\usepackage{url}
\usepackage{hyperref}
\usepackage{fontawesome}
\usepackage{marvosym}
\usepackage{acronym}
\usepackage{enumitem}
\usepackage{pdflscape}
\usepackage{listings}
\usepackage{siunitx}
\usepackage{icomma}
\usepackage{bigdelim}
\usepackage{isotope}
\setlength{\parindent}{0em} 

\renewcommand{\lstlistingname}{Python-Code}
\lstset{
	language=Python,
	frame=single,
	morekeywords={as,True,False},
	basicstyle=\footnotesize,
	keywordstyle=\color{blue}\bfseries,
	commentstyle=\color{red},
	stringstyle=\color{red},
	showstringspaces=false,
	showspaces=false,
	numbers=left,
	extendedchars=true,
	literate={ö}{{\"o}}1 {Ö}{{\"O}}1 {ä}{{\"a}}1 {Ä}{{\"A}}1 {ü}{{\"u}}1 {Ü}{{\"U}}1 {ß}{{\ss}}1}

\title{Introduction to Computational Physics  -- Exercise 5}
\author{Simon Groß-Bölting, Lorenz Vogel, Sebastian Willenberg}
\date{May 29, 2020}

\setcounter{section}{1}
% The beginning of the document...
\begin{document}
\maketitle

\section*{Numerical linear algebra methods: Tridiagonal matrices}
We consider the following tridiagonal $N\times N$ matrix equation:
\begin{equation}
	\underbrace{
	\begin{pmatrix}
	b_1 	& c_1		& 0 		& \cdots 	& 0 		\\
	a_2		& b_2 		& c_2		& \ddots 	& \vdots 	\\
	0		& a_3		& \ddots	& \ddots 	& 0 		\\
	\vdots 	& \ddots	& \ddots 	& \ddots	& c_{N-1} 	\\
	0 		& \cdots 	& 0 		& a_N 		& b_N 		\\
	\end{pmatrix}}_{\text{$=: M = (m_{ij})$}}
	\underbrace{\begin{pmatrix}
	x_1 \\ x_2 \\ x_3 \\ \vdots \\ x_{N-1} \\ x_N \\
	\end{pmatrix}}_{\text{$=: \vec{x} = (x_j)$}}
	=
	\underbrace{\begin{pmatrix}
	y_1 \\ y_2 \\ y_3 \\ \vdots \\ y_{N-1} \\ y_N \\
	\end{pmatrix}}_{\text{$=: \vec{y} = (y_i)$}}
\end{equation}
\textbf{Gauß elimination} is the method of choice when one is interested in the solution $\vec{x}$ of the linear set of equations $M\vec{x}=\vec{y}$, but does not need the information about the inverse matrix $M^{-1}$ (which we would get using the Gauß-Jordan method).
In this case, it is sufficient to convert the matrix $M$ into an upper (or lower) triangular
matrix $M^\prime$:
\begin{equation}
	M\vec{x} = \vec{y}
	\qquad\xlongrightarrow[]{\text{Gauß elimination}}\qquad
	M^\prime\vec{x} = \vec{y}^{\,\prime}
\end{equation}
If we have obtained such a system $M^\prime\vec{x}=\vec{y}^{\,\prime}$ with $M^\prime$ being in upper triangular form, i.e. $m^\prime_{ij}=0$ for $i>j$, then we can compute the solution vector $\vec{x}$ by \textbf{back substitution}:
\begin{equation}
	x_N = \frac{y^{\,\prime}_N}{m^\prime_{N,N}}
	\qquad
	x_{N-1} = \frac{1}{m^\prime_{N-1,N-1}}\left(y^{\,\prime}_{N-1}-m^\prime_{N-1,N-1}x_N\right)
	\qquad
	\cdots
\end{equation}
Or in general:
\begin{equation}
	x_i = \frac{1}{m^\prime_{ii}}\left(y^{\,\prime}_{i}-\sum_{j>i}m^\prime_{ij}x_j\right)
\end{equation}

The \textbf{Thomas algorithm} is a simplified form of Gauß elimination that can be used to solve tridiagonal systems of equation by creating a upper triangular form $M^\prime$ of the matrix $M=(m_{ij})$:
\begin{equation}
	\left.
	\begin{aligned}
		y_{i+1}		&\quad\longrightarrow\quad y_{i+1}-y_{i}\,\frac{m_{i+1,i}}{m_{ii}} \\
		m_{i+1,i+1} &\quad\longrightarrow\quad m_{i+1,i+1}-m_{i,i+1}\,\frac{m_{i+1,i}}{m_{ii}} \\
		m_{i+1,i} 	&\quad\longrightarrow\quad m_{i+1,i}-m_{ii}\,\frac{m_{i+1,i}}{m_{ii}}
	\end{aligned} 
	\qquad\right\rbrace\qquad\text{for $i=1,\,\ldots,\,N$}
\end{equation}
If we choose the parameters $a_i=-1$, $b_i=3$, $c_i=-1$ and $y_i=0.2$ $\forall i $ the solution vector beacomes the following:
\begin{equation}
	\vec{x}=
	\begin{pmatrix}
	0.12359551 \\
	0.17078652 \\
	0.18876404  \\
	0.19550562 \\
	0.19775281 \\
	0.19775281 \\
	0.19550562 \\
	0.18876404 \\
	0.17078652 \\
	0.12359551\\
	\end{pmatrix}
\end{equation}
To verify the solution we have implemented the Thomas Algoithm. The solution is the same. The relative difference is calculated in the following way:
\begin{equation}
	  	\Bigg\vert \frac{M\cdot \vec{x}-\vec{y}}{\vec{y}} \Bigg\vert
\end{equation}
We get the following relative difference from our solution to $\vec{y}$ :
\begin{equation}
	\Delta\vec{y}=
	\begin{pmatrix}
	
	\end{pmatrix}
\end{equation}

\newpage
\lstinputlisting[caption={Numerical solution of a tridiagonal system of equations },label={py:code}]{Tridiagonal-Matrices_Homework05_Gross-Boelting_Vogel_Willenberg.py}
 
\end{document}
