\documentclass[10pt, a4paper, nottitlepage]{article}

% Include packages 
\usepackage[left=2.5cm,right=2.5cm,top=2.5cm,bottom=2.5cm]{geometry}
\usepackage[utf8]{inputenc}
\usepackage[T1]{fontenc}
\usepackage[english]{babel}

\usepackage{float}
\usepackage{gensymb}
\usepackage{xcolor}
\usepackage{tikz}
\usepackage{graphicx}
\usepackage{lmodern}
\usepackage{amsmath,amsfonts,amssymb}
\usepackage{mathtools}
\usepackage{booktabs}
\usepackage{fancyhdr}
\usepackage{pdfpages}
\usepackage{extarrows}
\usepackage{subcaption}
\usepackage{url}
\usepackage{hyperref}
\usepackage{fontawesome}
\usepackage{marvosym}
\usepackage{acronym}
\usepackage{enumitem}
\usepackage{pdflscape}
\usepackage{listings}
\usepackage{siunitx}
%\usepackage{icomma}
\usepackage{bigdelim}
\usepackage{isotope}
\setlength{\parindent}{0em} 
\newcommand{\IU}{\mathrm{i}} 

\renewcommand{\lstlistingname}{Python-Code}
\lstset{
	language=Python,
	frame=single,
	morekeywords={as,True,False},
	basicstyle=\footnotesize,
	keywordstyle=\color{blue}\bfseries,
	commentstyle=\color{red},
	stringstyle=\color{red},
	showstringspaces=false,
	showspaces=false,
	numbers=left,
	extendedchars=true,
	literate={ö}{{\"o}}1 {Ö}{{\"O}}1 {ä}{{\"a}}1 {Ä}{{\"A}}1 {ü}{{\"u}}1 {Ü}{{\"U}}1 {ß}{{\ss}}1}

\title{Introduction to Computational Physics  -- Exercise 10}
\author{Simon Groß-Bölting, Lorenz Vogel, Sebastian Willenberg}
\date{July 3, 2020}

\setcounter{section}{1}
% The beginning of the document...
\begin{document}
\maketitle

%\section*{Probability Distribution Functions}
%Consider a probability distribution function $p(x)$ given in the domain $[0,\,a\rangle$ by
%\begin{equation}
%	p(x) = bx
%	\label{eq:probability-distribution}
%\end{equation}
%Assume that $\{r_i\}$ is a random set of numbers, distributed uniformly between $0$ and $1$.

%\begin{description}
%	\item[Task:] Give the proper value of $b$ as a function of $a$ such that the probability distribution function is properly normalized.
%	\item[Task:] Use the rejection method to make a set $\{x_i\}$ that obeys Eq. \eqref{eq:probability-distribution} for $a=0.5$.
%	\item[Task:] Make a histogram of the resulting numbers and check that the histogram indeed follows Eq. \eqref{eq:probability-distribution}, i.e. overplot Eq. \eqref{eq:probability-distribution}. Experiment with the size of the set (the number of random numbers drawn), to find out how large you have to make it to get (by eye) a reasonable fit.
%\end{description}

\section*{Determination of $\pi$ with Random Numbers}
\begin{description}
	\item[Task:] Compute the number $\pi$ using a rejection method with the function $f(x)=\sqrt{1-x^2}$, for $0\leq x\leq 1$. 
	\textit{Hint:} It is enough to use only one quadrant $x,\,f(x)>0$. Vary the number of
	random numbers (RNs) widely (orders of magnitude) and plot the accuracy of the result as a function of the
	number of RNs. Use logarithmic variables for the plot.
\end{description}

To determine $\pi$ numerically, we first use a so-called \textit{rejection method}: To do this, consider a quadrant $f(x)=\sqrt{1-x^2}$ ($0\leq x\leq 1$) with radius $r=1$ that lies within a square with the side lengths $r=1$. The following applies to the ratio of the areas of the quadrant and the square:
\begin{equation}
	\frac{A_\mathrm{c}}{A_\mathrm{s}} = \frac{\pi r^2}{4r^2} = \frac{\pi}{4}
\end{equation}
The goal is now to estimate the ratio of the areas in order to calculate $\pi$.
We can devise an algorithm that generates two random numbers $x_i,\,y_i\in [0,\,1)$ with $i=1,\,\dots,\,N$ (random coordinates from the square) and checks whether the coordinate fell into the quarter circle or not (rejection method): 
\begin{equation}
	\Bigl[\:\text{if}\quad y_i<\sqrt{1-x_i^2}\quad\Rightarrow\quad\text{``take''}\:\Bigr]
	\qquad\dot\lor\qquad
	\Bigl[\:\text{if}\quad y_i>\sqrt{1-x_i^2}\quad\Rightarrow\quad\text{``reject''}\:\Bigr]
\end{equation}
The value of $\pi$ is then approximately given by the ratio of the number $N_\mathrm{take}$ of points within the quadrant to the total number $N_\mathrm{total}$ of points:
\begin{equation}
	\pi\approx 4\:\frac{N_\mathrm{take}}{N_\mathrm{total}}
\end{equation}
To generate the random numbers, we use the function \texttt{numpy.random.rand}, which creates an array of the given shape and populates it with random samples from a uniform distribution over $[0,\,1)$.
Fig. \ref{fig:pi-determination} shows the idea/principle of the rejection method.
The accuracy (relative error compared to the ``true'' value of $\pi$) of the rejection method is shown in Fig. \ref{fig:accuracy}.\\\\
Another possibility for the numerical determination of $\pi$ is the so-called \textit{Monte Carlo integration}:
Monte Carlo integration is a technique for numerical integration using random numbers.
We want to integrate the function $f(x)$ on the interval $[a,\,b]$. The idea with Monte Carlo integration is to evaluate the function $f(x)$ on $N$ randomly selected points $x_i$ ($i=1,\,\dots,\,N$) (random sampling) in the interval $[a,\,b]$. In the case of equally distributed random numbers $x_i$, the value of the integral is simply the average of the function values $f(x_i)$:
\begin{equation}
	\int_{a}^{b}f(x)\,\mathrm{d}x \approx \frac{b-a}{N}\sum_{i=1}^{N}f(x_i) 
\end{equation}
The fact that this method works is due to the law of large numbers, i.e. $N\rightarrow\infty$.
We know that the integral of $f(x)=\sqrt{1-x^2}$ on the interval $0\leq x\leq 1$ gives a quarter of the area of the unit circle. The area of the unit circle (radius $r=1$) is $A_\mathrm{c}=\pi r^2=\pi$. With $N$ randomly selected values $x_i\in [0,\,1)$, we obtain the following approximation formula for $\pi$ using Monte Carlo integration:
\begin{equation}
	\frac{\pi}{4} = \int_{0}^{1}\sqrt{1-x^2}\,\mathrm{d}x \approx \frac{1}{N}\sum_{i=1}^{N}\sqrt{1-x_i^2}
	\qquad\Longrightarrow\qquad
	\pi\approx\frac{4}{N}\sum_{i=1}^{N}\sqrt{1-x_i^2}
\end{equation}
Using Monte Carlo integration, we get the following value for $\pi$:
\begin{equation}
\boxed{\:\text{Monte Carlo integration with $N=6\cdot 10^7$:}\qquad\quad
	\pi\approx\num{3.141557094}\:}
\end{equation}

\begin{figure}[t!]
	\centering
	\begin{subfigure}[t]{0.48\textwidth}
		\centering
		\includegraphics[width=\textwidth]{figures/RN-Rejection-Method_Pi-Determination.pdf}
		\caption{Illustration of the determination of $\pi$ using a\\rejection method with randomly selected points}
		\label{fig:pi-determination}
	\end{subfigure}
	\begin{subfigure}[t]{0.50\textwidth}
		\centering
		\includegraphics[width=\textwidth]{figures/RN-Rejection-Method_Accuracy-Analysis.pdf}
		\caption{Accuracy analysis of the determination of $\pi$ using a rejection method with randomly selected points}
		\label{fig:accuracy}
	\end{subfigure}
	\caption{Determination of $\pi$ with random numbers (rejection method)}
	\label{fig:rejection-method}
\end{figure}

\vspace{1cm}
\lstinputlisting[caption={Determination of $\pi$ using a rejection method and Monte Carlo integration},
label={py:code}]{Homework10_Random-Numbers_Gross-Boelting_Vogel_Willenberg.py}
\end{document}
