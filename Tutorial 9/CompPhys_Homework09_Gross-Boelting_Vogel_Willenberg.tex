\documentclass[10pt, a4paper, nottitlepage]{article}

% Include packages 
\usepackage[left=2.5cm,right=2.5cm,top=2.5cm,bottom=2.5cm]{geometry}
\usepackage[utf8]{inputenc}
\usepackage[T1]{fontenc}
\usepackage[english]{babel}

\usepackage{float}
\usepackage{gensymb}
\usepackage{xcolor}
\usepackage{tikz}
\usepackage{graphicx}
\usepackage{lmodern}
\usepackage{amsmath,amsfonts,amssymb}
\usepackage{mathtools}
\usepackage{booktabs}
\usepackage{fancyhdr}
\usepackage{pdfpages}
\usepackage{extarrows}
\usepackage{subcaption}
\usepackage{url}
\usepackage{hyperref}
\usepackage{fontawesome}
\usepackage{marvosym}
\usepackage{acronym}
\usepackage{enumitem}
\usepackage{pdflscape}
\usepackage{listings}
\usepackage{siunitx}
%\usepackage{icomma}
\usepackage{bigdelim}
\usepackage{isotope}
\setlength{\parindent}{0em} 
\newcommand{\IU}{\mathrm{i}} 

\renewcommand{\lstlistingname}{Python-Code}
\lstset{
	language=Python,
	frame=single,
	morekeywords={as,True,False},
	basicstyle=\footnotesize,
	keywordstyle=\color{blue}\bfseries,
	commentstyle=\color{red},
	stringstyle=\color{red},
	showstringspaces=false,
	showspaces=false,
	numbers=left,
	extendedchars=true,
	literate={ö}{{\"o}}1 {Ö}{{\"O}}1 {ä}{{\"a}}1 {Ä}{{\"A}}1 {ü}{{\"u}}1 {Ü}{{\"U}}1 {ß}{{\ss}}1}

\title{Introduction to Computational Physics  -- Exercise 9}
\author{Simon Groß-Bölting, Lorenz Vogel, Sebastian Willenberg}
\date{June 26, 2020}

\setcounter{section}{1}
% The beginning of the document...
\begin{document}
\maketitle


\section*{The Lorenz Attractor}
The Lorenz attractor problem is given by the following coupled set of differential equations:
\begin{align}
	\dot{x} &= -\sigma(x-y) \\
	\dot{y} &= rx-y-xz \\
	\dot{z} &= xy-bz
\end{align}
As discussed in the lecture, the fixed points are $\begin{pmatrix}0&0&0\end{pmatrix}$ for all $r$, and (for $r>1$) the points $C_{\pm}=\begin{pmatrix}\pm a_0&\pm a_0&r-1\end{pmatrix}$ with $a_0=\sqrt{b(r-1)}$. For the entire exercise, please use $\sigma=10$ and $b=8/3$.
The value of $r$ can be experimented with. When you create numerical solutions you can make plots in 2-D projection (e.g. in the $(x,y)$- or $(x,z)$-plane). You can also try a full 3-D plot.
\begin{description}
	\item[Task:] Solve numerically, using \texttt{rk4}, the above coupled set of equations for the values $r=0.5$, $1.17$, $1.3456$, $25.0$ and $29.0$. Choose the initial conditions near one of the fixed points: $C_{\pm}$ for $r>1$ and $\begin{pmatrix}0&0&0\end{pmatrix}$ for $r<1$. Explain the behavior, as much as possible, with the stability properties of the fixed points.
\end{description}

\begin{description}
	\item[Task:]  Determine the sequence $z_k$ for $r=26.5$, where $z_k$ is a local maximum in $z$
	on the solution curve after $k$ periods. Plot $z_{k+1}$ as a function of $z_k$.
	When sufficient points are there, connect the points. The resulting function $z_{k+1}=f(z_k)$ has an intersection with the diagonal $z_{k+1}=z_k$. It is a fixed point of the function $f(z_k)$. Is the slope $m$ of this function $>1$, $<-1$ or between $-1$ and $+1$?
	Notice: The theory of discrete maps says that there is no periodic solution if $|m|>1$. 
	So, in such a case we can deduce that this solution of the Lorenz system is not periodic.
\end{description}



%\vspace{1cm}
%\lstinputlisting[caption={Title},label={py:code}]{Homework09_Lorenz-Attractor_Gross-Boelting_Vogel_Willenberg.py}
\end{document}
=======
\end{document}
>>>>>>> e35b505317551e8d7b136ef6a86a98350e518195
