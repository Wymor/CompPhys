\documentclass[10pt, a4paper, nottitlepage]{article}

% Include packages 
\usepackage[left=2.5cm,right=2.5cm,top=2.5cm,bottom=2.5cm]{geometry}
\usepackage[utf8]{inputenc}
\usepackage[T1]{fontenc}
\usepackage[english]{babel}

\usepackage{float}
\usepackage{gensymb}
\usepackage{xcolor}
\usepackage{tikz}
\usepackage{graphicx}
\usepackage{lmodern}
\usepackage{amsmath,amsfonts,amssymb}
\usepackage{mathtools}
\usepackage{booktabs}
\usepackage{fancyhdr}
\usepackage{pdfpages}
\usepackage{extarrows}
\usepackage{subcaption}
\usepackage{url}
\usepackage{hyperref}
\usepackage{fontawesome}
\usepackage{marvosym}
\usepackage{acronym}
\usepackage{enumitem}
\usepackage{pdflscape}
\usepackage{listings}
\usepackage{siunitx}
\usepackage{icomma}
\usepackage{bigdelim}
\usepackage{isotope}
\setlength{\parindent}{0em} 

\renewcommand{\lstlistingname}{Python-Code}
\lstset{
	language=Python,
	frame=single,
	morekeywords={as,True,False},
	basicstyle=\footnotesize,
	keywordstyle=\color{blue}\bfseries,
	commentstyle=\color{red},
	stringstyle=\color{red},
	showstringspaces=false,
	showspaces=false,
	numbers=left,
	extendedchars=true,
	literate={ö}{{\"o}}1 {Ö}{{\"O}}1 {ä}{{\"a}}1 {Ä}{{\"A}}1 {ü}{{\"u}}1 {Ü}{{\"U}}1 {ß}{{\ss}}1}

\title{Introduction to Computational Physics  -- Exercise 4}
\author{Simon Groß-Bölting, Lorenz Vogel, Sebastian Willenberg}
\date{May 22, 2020}

\setcounter{section}{1}
% The beginning of the document...
\begin{document}
\maketitle

\section*{Neutrons in the gravitational field of the Earth}
The Numerov algorithm is an efficient approach to numerically solve the time-independet
Schrödinger equation. We want to use the Numerov algorithm to calculate the stationary states $\Psi(z)$ of neutrons in the gravitational field of the Earth.
For small changes in the vertical amplitude $z$ the potential can be expressed as $V(z)=mgz$ for $z\geq 0$. By placing a perfectly reflecting horizontal mirror at $z=0$ the potential becomes $V(z)=\infty$ for $z<0$. Neutrons that fall onto the mirror are reflected upwards, and so we only seek solutions for $z\geq 0$.
The stationary Schrödinger equation for this problem is given by 
\begin{equation}
\biggl\lbrace -\frac{\hbar^2}{2m}\frac{\partial^2}{\partial z^2}+mgz\biggr\rbrace\Psi(z) = E\Psi(z)\quad\text{,}
\end{equation}
with $E$ being the energy eigenvalue of the Hamilton operator.
We can rewrite this equation to:
\begin{equation}
\frac{\partial^2}{\partial z^2}\Psi(z) - \frac{2m}{\hbar^2}\left(E-mgz\right)\Psi(z) = 0
\qquad\Longleftrightarrow\qquad
\frac{\partial^2}{\partial z^2}\Psi(z) - \frac{2m^2g}{\hbar^2}\left(\frac{E}{mg}-z\right)\Psi(z) = 0
\end{equation}
By looking at the units of the factor in front of the bracket
\begin{equation}
\left[\frac{2m^2g}{\hbar^2}\right]
= 	\frac{\si{\square\kilogram\meter\per\square\second}}{\left(\si{\joule\second}\right)^2}
= 	\frac{\si{\kilogram\squared\meter}}{\si{\second\tothe{4}}}
\left(\frac{\si{\kilogram\square\meter}}{\si{\second\squared}}\right)^{-2}
=	\frac{\si{\kilogram\squared\meter}}{\si{\second\tothe{4}}}\:
\frac{\si{\second\tothe{4}}}{\si{\square\kilogram\meter\tothe{4}}}
=	\si{\per\cubic\meter}\quad\text{,}
\end{equation}
we find a suitable definition for the characteristic length scale $z_0$:
\begin{equation}
\frac{1}{z_0} := \left(\frac{2m^2g}{\hbar^2}\right)^{1/3}
\end{equation}
If we use the notation $x:=z/z_0$ and $\Psi(z):=\psi(x)/\sqrt{z_0}$, the second order partial derivative becomes:
\begin{equation}
\frac{\partial^2}{\partial z^2}\Psi(z)
=	\frac{1}{\sqrt{z_0}}\,\frac{1}{z_0^2}\,\frac{\partial^2}{\partial x^2}\psi(x)
=	\frac{1}{z_0^{5/2}}\,\frac{\partial^2}{\partial x^2}\psi(x)
\end{equation}
The differential equation then becomes:
\begin{equation}
\frac{1}{z_0^{5/2}}\,\frac{\partial^2}{\partial x^2}\psi(x) - 
\frac{1}{z_0^3}\,\frac{1}{\sqrt{z_0}}\left(\frac{E}{mg}-z\right)\psi(x) = 0
\qquad\Longleftrightarrow\qquad
\frac{\partial^2}{\partial x^2}\psi(x) - 
\frac{1}{z_0}\left(\frac{E}{mg}-z\right)\psi(x) = 0
\end{equation}
We now define the characteristic energy scale as $E_0:=mgz_0$.
The unit of $E_0$ is then given by:
\begin{equation}
\left[E_0\right] = \left[mgz_0\right]
=	\si{\kilogram\meter\per\square\second\meter}
= 	\si{\kilogram\square\meter\per\square\second}
=	\si{\joule}
\end{equation}

After this proper choice of length and energy units
\begin{equation}
x := \frac{z}{z_0} = z\left(\frac{2m^2g}{\hbar^2}\right)^{1/3}
\qquad\text{and}\qquad
\epsilon 	:= \frac{E}{E_0} = \frac{E}{mgz_0}
=  \frac{E}{mg}\left(\frac{2m^2g}{\hbar^2}\right)^{1/3}
\end{equation}
the Schrödinger equation for neutrons in the gravitational field of the Earth can be rewritten as:
\begin{equation}
\frac{\partial^2}{\partial x^2}\psi(x)+(\epsilon-x)\psi(x)
\qquad\Longleftrightarrow\qquad
\psi^{\prime\prime}(x) + (\epsilon-x)\psi(x) = 0
\label{eq:dimensionless-schroedinger}
\end{equation}
Eq. \eqref{eq:dimensionless-schroedinger} is a special variant of Sturm-Liouville differential equations of the type:
\begin{equation}
	y^{\prime\prime}(x)+k(x)y(x) = 0
\end{equation}
The Numerov algorithm is a highly accurate discretization method to solve such types of differential equations.
It is given by
\begin{equation}
	\left(1+\frac{1}{12}h^2k_{n+1}\right)y_{n+1}
	= 2\left(1-\frac{5}{12}h^2k_{n}\right)y_{n}
	  -\left(1+\frac{1}{12}h^2k_{n-1}\right)y_{n-1}+\mathcal{O}(h^6)
\end{equation}
and provides 6th order accuracy by using the three values $y_n$, $y_{n-1}$, $y_{n+1}$ with $k_i=k(x_i)$ and $y_i=y(x_i)$.
 
\begin{figure}[t!]
	\centering
	\includegraphics[width=0.7\textwidth]{figures/Asymptotic-Behavior.pdf}
	\caption{Asymptotic behavior of two solutions of the Numerov algorithm}
	\label{fig:asymptotic-behavior}	
\end{figure}

\subsection*{Asymptotic beavior of the solutions}
We are interested in the asymptotic behavior of the solution $\psi(x)$ for large $x$, i.e. whether it goes to positive or negative infinity.
By using the Numerov method, we numerically calculate the solutions for two different energy values $\epsilon$: One with positive and one with negative asymptotic behavior (see Fig. \ref{fig:asymptotic-behavior}). 


\subsection*{Energy eigenvalues}
The energy eigenvalues $E_n$ of Schrödinger's equation belong to normalizable eigenfunctions
with $\psi(x)\rightarrow 0$ for $x\rightarrow\infty$. 
It means that while varying $E_n$ (respectively $\epsilon_n$) from smaller to larger values, the function $\psi(x)$ for $x\rightarrow\infty$ changes sign. 
We use this property to determine the eigenvalues $E_n$ of the first three bound states to two decimals behind the comma (see Table \ref{tab:energy-eigenvalues} and Fig. \ref{fig:energy-eigenvalues}).

\begin{table}[h!]
	\centering
	\renewcommand{\arraystretch}{1.2}
	\caption{energy eigenvalues of the first three bound states}
	\vspace{-0.2cm}
	\begin{tabular}{ccc}
		\toprule
		quantum state $n$ & 
		dimensionless energy $\epsilon_n$ &
		energy eigenvalue $E_n$ [$\si{\pico\electronvolt}$] \\
		\midrule
		$1$ & $\num{2.34}$ & $\num{1.41}$ \\
		$2$ & $\num{4.09}$ & $\num{2.46}$ \\
		$3$ & $\num{5.52}$ & $\num{3.32}$ \\
		\bottomrule
	\end{tabular}
	\label{tab:energy-eigenvalues}
\end{table}


\newpage
\begin{figure}[t!]
	\centering
	\begin{subfigure}[t]{0.49\textwidth}
		\centering
		\includegraphics[width=\textwidth]{figures/Eigenvalues_Wavefunction.pdf}
		\caption{wavefunctions $\psi_n(x)$}
		\label{fig:eigenvalues-wavefucntion}
	\end{subfigure}
	\begin{subfigure}[t]{0.49\textwidth}
		\centering
		\includegraphics[width=\textwidth]{figures/Eigenvalues_Probability-Amplitude.pdf}
		\caption{probability amplitudes $|\psi_n(x)|^2$}
		\label{fig:eigenvalues-probability-amplitude}
	\end{subfigure}
	\caption{Wavefunctions and probability amplitudes of the quantum states of neutrons in the potential formed by the Earth's gravitational field and the horizontal mirror}
	\label{fig:energy-eigenvalues}
\end{figure}

\lstinputlisting[caption={Numerical solution of the Schrödinger equation using the Numerov method},label={py:code}]{Numerov-Algorithm_Homework04_Gross-Boelting_Vogel_Willenberg.py}
 
\end{document}
