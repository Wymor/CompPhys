\documentclass[10pt, a4paper, nottitlepage]{article}

% Include packages 
\usepackage[left=2.5cm,right=2.5cm,top=2.5cm,bottom=2.5cm]{geometry}
\usepackage[utf8]{inputenc}
\usepackage[T1]{fontenc}
\usepackage[english]{babel}

\usepackage{float}
\usepackage{gensymb}
\usepackage{xcolor}
\usepackage{tikz}
\usepackage{graphicx}
\usepackage{lmodern}
\usepackage{amsmath,amsfonts,amssymb}
\usepackage{mathtools}
\usepackage{booktabs}
\usepackage{fancyhdr}
\usepackage{pdfpages}
\usepackage{extarrows}
\usepackage{subcaption}
\usepackage{url}
\usepackage{hyperref}
\usepackage{fontawesome}
\usepackage{marvosym}
\usepackage{acronym}
\usepackage{enumitem}
\usepackage{pdflscape}
\usepackage{listings}
\usepackage[locale=DE]{siunitx}
\usepackage{icomma}
\usepackage{bigdelim}
\usepackage{isotope}
\setlength{\parindent}{0em} 
\lstset{literate=
  {²}{{$^2$}}1
  {η}{{$\eta$}}1
  {°}{{$\degree$}}1
  {±}{{$\pm$}}1
  {ö}{{\"o}}1 
  {ä}{{\"a}}1 
  {Ä}{{\"A}}1 
  {ü}{{\"u}}1 
  {Ü}{{\"U}}1 
  {ß}{{\ss}}1}
\usepackage{comment}

\definecolor{codegray}{rgb}{0.5,0.5,0.5}
\definecolor{backcolour}{rgb}{0.95,0.95,0.92}

\lstdefinestyle{mystyle}
    {backgroundcolor=\color{backcolour},   
    commentstyle=\color{cyan},
    keywordstyle=\color{orange},
    numberstyle=\tiny\color{codegray},
    stringstyle=\color{purple},
    basicstyle=\ttfamily\footnotesize,
    breakatwhitespace=false,         
    breaklines=true,                 
    captionpos=b,                    
    keepspaces=true,                 
    numbers=left,                    
    numbersep=5pt,                  
    showspaces=false,                
    showstringspaces=false,
    showtabs=false,                  
    tabsize=2}
\lstset{style=mystyle}

\title{Introduction to Computational Physics  -- Exercise 4}
\author{Simon Groß-Bölting, Lorenz Vogel, Sebastian Willenberg}
\date{May 22, 2020}

\setcounter{section}{1}
% The beginning of the document...
\begin{document}
\maketitle

The Numerov algorithm is an efficient approach to numerically solve the time-independet
Schrödinger equation. We want to use the Numerov algorithm to calculate the stationary states $\Psi(z)$ of neutrons in the gravitational field of the Earth.
For small changes in the vertical amplitude $z$ the potential can be expressed as $V(z)=mgz$ for $z\geq 0$. By placing a perfectly reflecting horizontal mirror at $z=0$ the potential becomes $V(z)=\infty$ for $z<0$. Neutrons that fall onto the mirror are reflected upwards, and so we only seek solutions for $z\geq 0$.
The stationary Schrödinger equation for this problem is given by 
\begin{equation}
\biggl\lbrace -\frac{\hbar^2}{2m}\frac{\partial^2}{\partial z^2}+mgz\biggr\rbrace\Psi(z) = E\Psi(z)\quad\text{,}
\end{equation}
with $E$ being the energy eigenvalue of the Hamilton operator.
We can rewrite this equation to:
\begin{equation}
\frac{\partial^2}{\partial z^2}\Psi(z) - \frac{2m}{\hbar^2}\left(E-mgz\right)\Psi(z) = 0
\qquad\Longleftrightarrow\qquad
\frac{\partial^2}{\partial z^2}\Psi(z) - \frac{2m^2g}{\hbar^2}\left(\frac{E}{mg}-z\right)\Psi(z) = 0
\end{equation}
By looking at the units of the factor in front of the bracket
\begin{equation}
\left[\frac{2m^2g}{\hbar^2}\right]
= 	\frac{\si{\square\kilogram\meter\per\square\second}}{\left(\si{\joule\second}\right)^2}
= 	\frac{\si{\kilogram\squared\meter}}{\si{\second\tothe{4}}}
\left(\frac{\si{\kilogram\square\meter}}{\si{\second\squared}}\right)^{-2}
=	\frac{\si{\kilogram\squared\meter}}{\si{\second\tothe{4}}}\:
\frac{\si{\second\tothe{4}}}{\si{\square\kilogram\meter\tothe{4}}}
=	\si{\per\cubic\meter}\quad\text{,}
\end{equation}
we find a suitable definition for the characteristic length scale $z_0$:
\begin{equation}
\frac{1}{z_0} := \left(\frac{2m^2g}{\hbar^2}\right)^{1/3}
\end{equation}
If we use the notation $x:=z/z_0$ and $\Psi(z):=\psi(x)/\sqrt{z_0}$, the second order partial derivative becomes:
\begin{equation}
\frac{\partial^2}{\partial z^2}\Psi(z)
=	\frac{1}{\sqrt{z_0}}\,\frac{1}{z_0^2}\,\frac{\partial^2}{\partial x^2}\psi(x)
=	\frac{1}{z_0^{5/2}}\,\frac{\partial^2}{\partial x^2}\psi(x)
\end{equation}
The differential equation then becomes:
\begin{equation}
\frac{1}{z_0^{5/2}}\,\frac{\partial^2}{\partial x^2}\psi(x) - 
\frac{1}{z_0^3}\,\frac{1}{\sqrt{z_0}}\left(\frac{E}{mg}-z\right)\psi(x) = 0
\qquad\Longleftrightarrow\qquad
\frac{\partial^2}{\partial x^2}\psi(x) - 
\frac{1}{z_0}\left(\frac{E}{mg}-z\right)\psi(x) = 0
\end{equation}
We now define the characteristic energy scale as $E_0:=mgz_0$.
The unit of $E_0$ is then given by:
\begin{equation}
\left[E_0\right] = \left[mgz_0\right]
=	\si{\kilogram\meter\per\square\second\meter}
= 	\si{\kilogram\square\meter\per\square\second}
=	\si{\joule}
\end{equation}

After this proper choice of length and energy units
\begin{equation}
x := \frac{z}{z_0} = z\left(\frac{2m^2g}{\hbar^2}\right)^{1/3}
\qquad\text{and}\qquad
\epsilon 	:= \frac{E}{E_0} = \frac{E}{mgz_0}
=  \frac{E}{mg}\left(\frac{2m^2g}{\hbar^2}\right)^{1/3}
\end{equation}
the Schrödinger equation for neutrons in the gravitational field of the Earth can be rewritten as:
\begin{equation}
\frac{\partial^2}{\partial x^2}\psi(x)+(\epsilon-x)\psi(x)
\qquad\Longleftrightarrow\qquad
\psi^{\prime\prime}(x) + (\epsilon-x)\psi(x) = 0
\end{equation}


\end{document}
